\documentclass[letterpaper]{article}

\usepackage[letterpaper, total={7in, 8in}]{geometry}
\usepackage[spanish]{babel}
\usepackage[utf8]{inputenc}
\usepackage{fancyhdr}
\usepackage{graphicx}
\usepackage{booktabs}
\usepackage{multirow}
\usepackage{hyperref}

\fancyhf{}
\pagestyle{fancy}

\hypersetup{
    colorlinks=true,
    linkcolor=blue,
    filecolor=magenta,      
    urlcolor=cyan,
    pdftitle={Proyecto cohetería computacional IIS 2022},
}

\lhead{Versión 1.0}
\rhead{Cohetería Computacional}

\title{Introducción a cohetería computacional, segundo semestre del 2022, información del curso}
\author{Luis Ross Lépiz}

\begin{document}

\maketitle
\thispagestyle{empty}
\begin{center}
    Versión 1.0
\end{center}

\begin{figure}[b]
    \centering
    \includegraphics[width=0.8\textwidth]{img/logo.png}
\end{figure}
\clearpage

\tableofcontents
\clearpage

\section{Descripción del proyecto}
El proyecto de computacional evaluará los siguientes tres elementos a construir y programar:
\begin{itemize}
    \item Computadora de vuelo que resida dentro del cohete y esté activa en todo momento del vuelo, 
        tomando datos que puedan ser analizados.
    \item Aplicación que el equipo en tierra pueda usar durante el vuelo para monitorear y controlar la computadora de vuelo.
    \item Sistema que facilite la recuperación del cohete luego del aterrizaje, esto puede ser 
        integrado a la computadora de vuelo y la aplicación de tierra, o bien puede ser un sistema totalmente aparte.
\end{itemize}

\subsection{Computadora de vuelo}
Debe ser capaz de monitorear el vuelo en tiempo real una o más variables del vuelo, estas quedan a 
decisión del equipo y preferiblemente que puedan justificar por qué realizaron la elección (costos, 
análisis de datos, archivación de datos, disponibilidad de sensores, entre otros), algunas variables 
a considerar pueden ser:
\begin{itemize}
    \item Velocidad.
    \item Aceleración.
    \item Altitud.
    \item Temperatura (en zona de interés).
    \item Vibraciones.
    \item etc...
\end{itemize}

La computadora usará un microcontrolador (queda a elección del equipo cuál usar), el mismo energizado 
mediante una batería con suficiente capacidad para el tiempo de uso durante los lanzamientos (se
recomienda la colocación de un interruptor en el exterior del cohete, así permitiendo sólo activar 
el sistema cuando se tenga que utilizar y no gastar batería cuando no se use).

La computadora de vuelo deberá permitir la comunicación remota para transmisión de datos y recepción 
de comandos, por lo tanto es importante la elección de un protocolo de comunicación apto para la
cantidad de datos que se desean transmitir y la distancia máxima que se requiera cubrir.

\subsection{Aplicación de control}
Permite monitorear en tiempo real los valores desde tierra, esta aplicación puede ser una interfáz 
gráfica (GUI), una aplicación de terminal (CLI) o bien interfaz de texto (TUI), en cualquier 
lenguaje o plataforma (Windows, Linux, macOS, Android, iOS, microcontrolador, interfaz web, etc)
según sea el gusto de los integrantes.

La aplicación deberá conectarse en tiempo real con la computadora de vuelo, por lo tanto el
protocolo de comunicación usado para la computadora de vuelo deberá también ser considerado para la
plataforma donde se tendrá esta aplicación pues deberá tener compatibilidad con esta.

Los datos recibidos desde la computadora de vuelo deberán ser almacenados en tiempo real, esto
quiere decir que justo en el momento que son recibidos y procesados se deben almacenar en la memoria
del dispositivo, lo que implica que nunca se pierdan los datos recibidos del vuelo, incluso si se 
terminara la comunicación de manera abrupta, la aplicación se cerrara, o el dispositivo se apagara,
siempre quedarían los datos guardados en la memoria para análisis futuro.

\subsection{Sistema de recuperación}
Se promueve la creatividad en este apartado, cada grupo podrá elegir un sistema que facilite la
recuperación siempre y cuando sea seguro de implementar y primero sea consultado con el profesor del
curso.

La idea de este sistema consiste en que desde el equipo en tierra, se le pueda enviar al cohete una
señal o activar un sistema que indique dónde aterrizó el cohete, esto preferiblemente que con una
sóla activación se mantenga generando una señal fácil de encontrar en medio de ambientes
complicados, como zacate alto, huecos, detrás de árboles, etc.

Algunos ejemplos que pueden considerar son los siguientes, pero se les insta a ser creativos:
\begin{itemize}
    \item Bocina que emita un sonido repetidas veces luego de presionar un botón en la interfaz
        gráfica de la aplicación de control.
    \item Botón en un dispositivo de mano que emita constantemente la distancia aproximada entre
        el cohete y el dispositivo (analizando la intensidad de señal de conexión).
    \item Sistema que libere humo de una manera totalmente segura que nunca puede salir mal e
        incendiar el cohete, al escribir un comando en la aplicación de terminal.
    \item etc...
\end{itemize}

\section{Evaluación del proyecto}
La distribución de puntos del proyecto se muestra a continuación, este se evalúa en base 100 y será
la misma nota para todos los integrantes del grupo de computacional, esto no es igual a la nota
final del curso pues esta también dependerá de otros factores explicados más adelante en la sección 
\ref{nota_final}.

\begin{table}[h]
    \centering
    \begin{tabular}{cccc}
        \toprule
        \multicolumn{2}{c}{Elementos a evaluar} & Valor individual (pts) & Valor total (pts) \\
        \midrule
        \bf{Computadora de vuelo} & & - & \multirow{5}{*}{40} \\
        \multicolumn{2}{l}{Microcontrolador apto para la tarea} & 15 & \\
        \multicolumn{2}{l}{Uso de sensores y/o actuadores}      & 10 & \\
        \multicolumn{2}{l}{Comunicación remota funcional}       & 10 & \\
        \multicolumn{2}{l}{Batería de fácil acceso}             & 5 & \\
        \midrule
        \bf{Aplicación de control} & & - & \multirow{4}{*}{40} \\
        \multicolumn{2}{l}{Interfaz de usuario funcional}       & 20 & \\
        \multicolumn{2}{l}{Recepción de datos en tiempo real}   & 10 & \\
        \multicolumn{2}{l}{Almacenamiento en tiempo real}       & 10 & \\
        \midrule
        \bf{Sistema de recuperación} & & - & \multirow{4}{*}{20} \\
        \multicolumn{2}{l}{Activación remota exitosa}           & 8 & \\
        \multicolumn{2}{l}{Efectividad para encontrar cohete}   & 8 & \\
        \multicolumn{2}{l}{Creatividad de la solución}          & 4 & \\
        \midrule
        \bf{Total} & & & 100 \\
        \bottomrule
    \end{tabular}
\end{table}

\clearpage

%\section{Nota final del curso}
%\label{nota_final}
%Como se indicó previamente, la nota final del curso para cada estudiante constará tanto de la nota
%grupal del proyecto, como de otros valores, estos se detallan a continuación.
%\begin{table}[h]
%    \centering
%    \begin{tabular}{cc}
%        \toprule
%            Item a evaluar & Valor del item [\%] \\
%        \midrule
%            Proyecto computacional & 55 \\
%            Coevaluación proyecto  & 10 \\
%            Tareas computacional   & 35 \\
%        \bottomrule
%    \end{tabular}
%\end{table}
%
%Todo estudiante que obtenga una nota de 70 será considerado aprobado del curso de cohetería
%computacional, esto significa que podrá participar en los proyectos de TECSpace como integrante y
%recibirá un reconocimiento por haber aprobado el curso, el cual puede ser usado para demostrar
%experiencia y dedicación.

\end{document}

