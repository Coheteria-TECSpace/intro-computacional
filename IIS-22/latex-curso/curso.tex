\documentclass[letterpaper]{article}

\usepackage[letterpaper, total={7in, 8in}]{geometry}
\usepackage[spanish]{babel}
\usepackage[utf8]{inputenc}
\usepackage{fancyhdr}
\usepackage{graphicx}
\usepackage{booktabs}
\usepackage{multirow}
\usepackage{hyperref}

\fancyhf{}
\pagestyle{fancy}

\hypersetup{
    colorlinks=true,
    linkcolor=blue,
    filecolor=magenta,      
    urlcolor=cyan,
    pdftitle={Proyecto cohetería computacional IIS 2022},
}

\lhead{Versión 1.0}
\rhead{Cohetería Computacional}

\title{Introducción a cohetería computacional, segundo semestre del 2022, información del curso}
\author{Luis Ross Lépiz}

\begin{document}

\maketitle
\thispagestyle{empty}
\begin{center}
    Versión 1.0
\end{center}

\begin{figure}[b]
    \centering
    \includegraphics[width=0.8\textwidth]{img/logo.png}
\end{figure}
\clearpage

\tableofcontents
\clearpage

\section{Descripción general}
Introducción a cohetería computacional tiene como intención brindarle a los estudiantes una idea
sobre el conocimiento y trabajo que va detrás de diseñar un sistema computacional para una misión de
cohetería, esto desde los aspectos de electrónica, programación y un poco de organización de trabajo
en equipo.

El estudiante que disfrute de esta área y aproveche el curso tendrá herramientas útiles para todo
momento en programación así como conocimiento general sobre cómo funcionan todos los equipos que nos
rodean día a día desde el celular en nuestras manos, hasta los chips que envían información desde
satélites y aparatos espaciales en el espacio exterior.

\section{Objetivos}
\subsection{Objetivo General}
Brindar el conocimiento necesario de programación, electrónica y telemetría para llevar a cabo una
misión de cohetería exitosa, donde se diseñe una computadora para el cohete, una aplicación para
control de misión y un sistema de recuperación que facilite la misma, así aportando a la misión en
toda etapa.

\subsection{Objetivos Específicos}
\begin{enumerate}
    \item Enseñar los fundamentos detrás de los lenguajes de programación que se usan en la
        actualidad, así como un trasfondo histórico sobre los mismos.
    \item Aprender a programar en lenguajes de alto nivel bastante comúnes como lo son Python y
        Arduino, para poder programar tanto en computadoras comúnes como en microcontroladores.
    \item Diseñar circuitos básicos con conceptos simples de corriente continua y el uso de
        microcontroladores programables para automatizar tareas.
    \item Usar sensores y actuadores según sea necesario para las soluciones de electrónica que se
        consideren necesarias en los circuitos creados.
    \item Conocer la teoría de telemetría y poder aplicarla para realizar conexiones inalámbricas y
        transmisión de datos a distancia en los dispositivos elegidos.
    \item Programar una aplicación pensada para el uso de un usuario promedio, con funcionalidades
        útiles o llamativas antes, durante y después del lanzamiento y aterrizaje del cohete.
    \item Trabajar en conjunto para diseñar un sistema completo e inter funcional de electrónica
        y programación.
    \item Ganar conocimiento de programación que pueda ser aplicado a cualquier lenguaje que se
        encuentren en un futuro en su carrera profesional.
\end{enumerate}

\clearpage
\section{Contenidos}
\subsection{Conocimiento general sobre qué es cohetería computacional}
\begin{itemize}
    \item Hardware a lo largo de la historia.
    \item Software a lo largo de la historia.
    \item Computación en el espacio (NASA y SpaceX).
    \item Objetivos de cohetería computacional en TECSpace.
    \item Tecnologías que usamos en TECSpace.
    \item Ejemplos de proyectos existentes y viables.
\end{itemize}
\subsection{Fundamentos tras los lenguajes de programación y sus diferencias}
\begin{itemize}
    \item Diferencias entre lenguajes de alto y bajo nivel.
    \item Paradigmas de progrmación y sus usos.
    \item Diferencias entre lenguajes compilados e interpretados.
    \item Compatibilidad de sistemas operativos.
    \item Compatibilidad de navegadores web.
    \item Compatibilidad de dispositivos (arquitecturas).
    \item Pseudocódigo y diagramas de flujo.
\end{itemize}
\subsection{Taller básico de programación en Python con enfoque al aprendizaje de otros lenguajes}
\begin{itemize}
    \item Qué son variables detalles y cómo usarlas.
    \item Instaciación de variables vs declaración de valores.
    \item Utilidad de funciones, cómo usarlas y sus partes.
    \item Código estructurado para manejo de pruebas lógicas.
    \item Ciclos de código do while, y do for.
    \item Listas en Python sus usos y cómo se diferencian de arrays.
    \item Cómo importar código de otros archivos en Python (módulos).
    \item Lectura y escritura de datos en archivos de texto I/O.
    \item Cómo facilitar el aprendizaje u entendimiento de lenguajes que no conocemos.
\end{itemize}
\subsection{Introducción a microcontroladores y taller básico de electrónica con Arduino}
\subsection{Introducción a telemetría y taller básico de comunicaciones remotas}

\clearpage

\section{Evaluación}
La nota final del curso para cada estudiante constará tanto de la nota
grupal del proyecto, como de otros valores, estos se detallan a continuación.
\begin{table}[h]
    \centering
    \begin{tabular}{cc}
        \toprule
            Item a evaluar & Valor del item [\%] \\
        \midrule
            Proyecto computacional & 55 \\
            Coevaluación proyecto  & 10 \\
            Tareas computacional   & 35 \\
        \bottomrule
    \end{tabular}
\end{table}

Todo estudiante que obtenga una nota de 70 será considerado aprobado del curso de cohetería
computacional, esto significa que podrá participar en los proyectos de TECSpace como integrante y
recibirá un reconocimiento por haber aprobado el curso, el cual puede ser usado para demostrar
experiencia y dedicación.

\end{document}

